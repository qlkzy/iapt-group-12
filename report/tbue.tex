\section{Task-Based User Evaluation}

\subsection{Methodology}

According to Marieke McCloskey in an article she wrote recently called
``Turn User Goals into Task Scenarios for Usability
Testing''\todo{reference}, you should use the goals of your system to
guide you when creating tasks for task based evaluations. This idea
resonated with our group; it follows that the best place to start when
designing a good test scenario would be to ensure that it adequately
tests that the user can achieve the core goals that they need to. As
such we decided to first begin by defining the goals we hope to test.

\tbueGoal{Being able to find a recipe} To do this they must be able
to browse content of the site in some way and be able to utilise
search and filters to find a specific recipe.

\tbueGoal{Viewing a recipe} To do this they must be able to
effectively manipulate view types and acquire all other information
needed to carry out the recipe.

\vspace{5mm}
Marieke also defined three key principles to keep in mind to when
creating tasks that we endeavoured to follow:

\tbuePrinciple{Make the task relate to real life} She argues that making
the task feel real makes the user perform more realistically: if you
can suspend the user's disbelief, you will get more reliable test
data. Even if you ask the user to do something they usually wouldn't,
if you can flesh out the scenario enough you can achieve this effect.

\tbuePrinciple{Make them actionable} She describes it as important to make
the user actually use the system, it seems obvious but if the task is
phrased badly the user could stop trying to actually use the system
and instead just describe what they would do to the tester, which
doesn’t provide any useful data.

\tbuePrinciple{Avoid clues in task step} (Avoid using terms used on the site).
Obviously, if the task contains phrases the user sees on the site they
will gravitate towards them, even if that wouldn't be their normal
interaction with the site.


So, ensuring we followed these principles, and that our tasks
encompass all the identified goals, the following tasks and scenarios
were created:

\tbueTask{Task 1}{
  You are an experienced cook, you are looking for
  a challenge and would therefore only like to cook something that is
  hard to make.
}{
  Find the list of available recipes for main
  dishes that will pose a significant challenge in preparing.
}

\tbueTask{Task 2}{
  You love vanilla slice, but have no idea how to
  make it.
}{
  Find a Vanilla Slice recipe and view it in
  its most verbose form.
}

\tbueTask{Task 3}{
  You are an experienced cook and a vegetarian, you only have
  about 30 minutes before you have to leave for work.
}{
  Instructions: Find the list of vegetarian dishes that can be prepared
  within 20 minutes.
}

\tbueTask{Task 4}{
  You are entering a pie making competition.
}{
  Bring up a list of all pie recipes.
}

\tbueTask{Task 5}{
  You were a novice user but have now used the website quite a
  lot. You are browsing a recipe and decide you do not need the step by
  step instructions anymore.
}{
  Change the default recipe view.
}

\subsubsection{Who to use to test?}

As described by By Deborah Hinderer Sova and Jakob Nielsen in their
report ``How to Recruit Participants for Usability
Studies''\todo{reference} the cardinal rule for recruiting is ``know
thy users''. We have already gathered this information as part of our
user-centered design process, to make our personae. To get the most
out of our task based evaluation we will need to ensure our 8 test
candidates represent a good mixture of our predicted user
base. Obviously as students getting other students will be easy, but
recruiting computer illiterate testers will prove more
difficult. Thankfully, one of our team members was able to get their
parents to help out; less than ideal, as that particular user group
will be under represented, but probably sufficient.

Nielsen also stresses the importance of focusing on what users do, not
just on what they say: they may say they didn't struggle to find an
element on the page, but their eye and mouse movements tell a
different story. In the absence of the necessary hardware and software
to monitor eye and mouse movements we instead opted to carry out the
tests in pairs. This way one conductor can focus on the user and the
other on the screen.

However, we can still use what the user has to say about the system,
and we used the same rating system as in the CHE. This provides a
solid metric with which we can identify key problems with our system.

\subsection{TBUE Data}

The results of the task-based user evaluation are summarised in the
following table. As noted, severity ratings are the same as those used
for the CHE\todo{reference}.

\begin{tbuetable}{Task 1: Find challenging recipes}

  1 & Expected filter to apply after pressing enter, but filter button
  present & 1 \\

  1 & Felt filter button could stand out more, felt it was small but
  also said it was easy to find as it was where they expected it to be &
  0 \\

  1 & Skipped over welcome page tutorial, then wasn't sure which view
  type to use for experienced cook, cycled through all the views until
  they found one they liked. & 2 \\

  2 & Read welcome page thoroughly, commented not all information was
  needed, felt that it slowed them down a bit & 1 \\

  2 & Completed task with no problems, initially drawn to list of
  recipes but found filters quickly & 0 \\

\end{tbuetable}

\begin{tbuetable}{Task 2: View a Vanilla Slice recipe in its most
    verbose form}

  3 & Not obvious what the different views are---descriptions not very
  obvious, wasn’t sure what the most verbose view is, content layout is
  fine & 1 \\

  3 & Ingredients list not formatting fractions properly, hard to read
  but can still make out values & 1 \\

  3 & Asked to choose default view, just want to see recipes, slows
  user down, welcome section too prominent ignore the search box & 2
  \\

  3 & Cosmetic issue title not being capitalised & 0 \\

  3 & Search bar not obvious, user scrolled first before choosing to
  search, but they did find the recipe complained that it took a bit
  longer & 2 \\

\end{tbuetable}


\begin{tbuetable}{Task 3: Find all vegetarian dishes that can be
    prepared in less than 20 minutes}

  3 & Search vegetarian, didn’t return anything, filters are there so
  can recover & 2 \\

  4 & Felt welcome page was not relevant to their task or scenarion,
  slowed them down & 1 \\

  4 & Filter by vegetarian gives nothing, no information to suggest if
  search return was empty or website errored, assumed website had broken
  and stopped & 4 \\

\end{tbuetable}

\begin{tbuetable}{Task 4: Find all pie recipes}

  5 & Pie not left in the search bar, even though title of page says
  ``you searched for pie'' inconsistency with pie not displayed in top
  search bar & 1 \\

  6 & Completed task with no problems & 0 \\

\end{tbuetable}

\begin{tbuetable}{Task 5: Change the default recipe view}

  7 & No problems reported, went back to page they orginally saw & 0 \\

  8 & Nothing on page to show how they could change it, guess its in
  help, help doesn’t help doesn’t mention default view. Doesn't know
  what th default view is or how to change it. & 4 \\

  8 & Stumbled upon answer by going back to home page, and going through
  first user flow & 4 \\

\end{tbuetable}


For the sake of brevity, we will not reproduce all of the notes
associated with the task-based user evaluation here. However, a couple
of the cases were particularly interesting from a usability point of
view; what follows is a more in-depth description of the users'
interactions while completing these tasks.

\paragraph{Task: find vanilla slice recipe and view it in its most verbose form.}

The user successfully completed the task by navigating the tutorial
and then using the search bar to find the desired recipe. Afterwards
he did express some concern over finding the most `verbose' recipe
view described by the task. He further elaborated that whilst he was
given a summary of all the view types in the tutorial he failed to
decipher which view the task was referring to based on the tutorial
text. Perhaps work needs to be done to change the tutorial text to
make it more obvious, but the user did confess to not reading the
tutorial particularly thoroughly.

The user did state he had no issues with the layout of the tutorial,
the text was obvious and easy to read, but it seems the user was more
focused on finding the recipe itself; although the tutorial presented
relevant information, the user described it as `slowing him
down'. Further elaboration reveals this user would have been happy to
discover the different views on his own, but he did state he probably
would want to get this information at a later time. The user gave this
a problem rating of 2, but they also claimed a large part of the fault
lay in the ambiguity between the task and the website and so their
rating may be weighted a bit heavily. From our point of view this
purposeful disconnect provided valuable additional information on how
the tutorial system is perceived by this user.

\paragraph{Task: Change the default recipe view}

The user remembered that they were given the opportunity to change
default view from the tutorial on the home page. The opted to go
straight back to the home page and change the default from there. They
did however concede that they did not expect the tutorial to show up
every time but opted for that route after a quick scan of the recipe
page showed no obvious way to change the default.

Another user, however struggled with the task: they spent a long time
on the recipe page particularly around the view changing buttons, to
no avail. They then attempted to go to help, finding that nothing
there was helpful in fulfilling the task. Eventually they stumbled
upon the home screen and were presented with the tutorial. They
strongly disliked the flow they took and it took a long time to
complete the task, so they rated the severity as 4, a very serious
problem.

\subsection{Problems and subsequent changes}

The task-based user evaluation highlighted a number of usability
issues and bugs in the website. The two highest-severity issues
highlighted by the evaluation were:

Regarding search:
\begin{quote}
  \footnotesize
  Filter by vegetarian gives nothing, no information to suggest if
  search return was empty or website errored, assumed website had
  broken and stopped.
\end{quote}

and, regarding choosing the default recipe view:
\begin{quote}
  \footnotesize
  Nothing on page to show how they could change it, guess its in help,
  help doesnt help doesnt mention default view. Doesn’t know what the
  default view is or how to change it.
\end{quote}

Of these, the issue with the filters was a straightforward bug,
whereas the issue with default view selection was a serious usability
problem. In addition, the mechanics associated with varying
presentations of recipes are a core piece of functionality---the
`killer feature', in some sense. We will therefore present the issues
with and redesign of the functionality around default recipe selection.

In the website prior to the task-based evaluation (hereafter `the
original'), there was no means to change the default view from the
recipe view page\todo{reference}. It was possible to change the
default view, but only by going all the way back to the entry point of
the application and following through the welcome
process\todo{reference}.

This was extremely surprising to the users, unless they were given the
task of changing the default view while still on the welcome page
(whereas we would expect users to want to change the default view
while browsing and viewing recipes).

To address this issue, we considered the context for changing the
default view, and the information needed to do so, taking advantage of
our observations of the users during the evaluation.

\todo{Come up with more synonyms for selecting things}

Changing the view for recipes is strongly associated with actually
viewing the recipes, and changing the default view is naturally
associated with the rest of the view-changing machinery. In
redesigning this functionality, we took account of this context by
making the control to change the default view a `sibling' to the
existing controls for switching between views.

When changing the default view, the user needs to be able to evaluate
the different views and choose one. The key information they need for
this is how the view will appear in use. In the redesign, we ensured
that the user would have this information by providing a control to
make the \emph{current} view the default, rather than, say, a separate
choice from a list of alternatives. This allows the user to explore
the site and try out views until they find a presentation they are
comfortable with, at which point they can make it the default. This
follows a pattern seen in a number of other pieces of software (that
is `make \emph{this} the default' rather than `\emph{select} a
default'), hopefully allowing the user to make use of their previous
experience.

In the redesign, we addressed these considerations by adding a new
control element underneath the view-changer buttons. When the current
view is the default, this is uninteractive, and tells the user that
the selected view is the default\todo{reference}. When the current
view is \emph{not} the default, the control offers the user the
opportunity to set the current view as the default\todo{reference}.

To further improve the discoverability of this feature, the
previously-implemented pop-up help functionality was extended to
include an explanation of this feature.

Allowing the user to choose the default view from the recipe page also
made it possible to hide the `Welcome' screen, with \emph{its} option
to choose the default view, after the user first sees it. This
addresses the user frustration, noted above, that it was slightly
awkward to get back to the list of recipes.
