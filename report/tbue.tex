\section{Task-Based User Evaluation}

\subsection{Methodology}

According to Marieke McCloskey in an article she wrote recently called
``Turn User Goals into Task Scenarios for Usability
Testing''\todo{reference}, you should use the goals of your system to
guide you when creating tasks for task based evaluations. This idea
resonated with our group; it follows that the best place to start when
designing a good test scenario would be to ensure that it adequately
tests that the user can achieve the core goals that they need to. As
such we decided to first begin by defining the goals we hope to test.

\tbueGoal{Being able to find a recipe} To do this they must be able
to browse content of the site in some way and be able to utilise
search and filters to find a specific recipe.

\tbueGoal{Viewing a recipe} To do this they must be able to
effectively manipulate view types and acquire all other information
needed to carry out the recipe.

\vspace{5mm}
Marieke also defined three key principles to keep in mind to when
creating tasks that we endeavoured to follow:

\tbuePrinciple{Make the task relate to real life} She argues that making
the task feel real makes the user perform more realistically: if you
can suspend the user's disbelief, you will get more reliable test
data. Even if you ask the user to do something they usually wouldn't,
if you can flesh out the scenario enough you can achieve this effect.

\tbuePrinciple{Make them actionable} She describes it as important to make
the user actually use the system, it seems obvious but if the task is
phrased badly the user could stop trying to actually use the system
and instead just describe what they would do to the tester, which
doesn’t provide any useful data.

\tbuePrinciple{Avoid clues in task step} (Avoid using terms used on the site).
Obviously, if the task contains phrases the user sees on the site they
will gravitate towards them, even if that wouldn't be their normal
interaction with the site.


So, ensuring we followed these principles, and that our tasks
encompass all the identified goals, the following tasks and scenarios
were created:

\tbueTask{Task 1}{
  You are an experienced cook, you are looking for
  a challenge and would therefore only like to cook something that is
  hard to make.
}{
  Find the list of available recipes for main
  dishes that will pose a significant challenge in preparing.
}

\tbueTask{Task 2}{
  You love vanilla slice, but have no idea how to
  make it.
}{
  Find a Vanilla Slice recipe and view it in
  its most verbose form.
}

\tbueTask{Task 3}{
  You are an experienced cook and a vegetarian, you only have
  about 30 minutes before you have to leave for work.
}{
  Instructions: Find the list of vegetarian dishes that can be prepared
  within 20 minutes.
}

\tbueTask{Task 4}{
  You are entering a pie making competition.
}{
  Bring up a list of all pie recipes.
}

\tbueTask{Task 5}{
  You were a novice user but have now used the website quite a
  lot. You are browsing a recipe and decide you do not need the step by
  step instructions anymore.
}{
  Change the default recipe view.
}

\subsection{Who to use to test?}

As described by By Deborah Hinderer Sova and Jakob Nielsen in their
report ``How to Recruit Participants for Usability
Studies''\todo{reference} the cardinal rule for recruiting is ``know
thy users''. We have already gathered this information as part of our
user-centered design process, to make our personae. To get the most
out of our task based evaluation we will need to ensure our 8 test
candidates represent a good mixture of our predicted user
base. Obviously as students getting other students will be easy, but
recruiting computer illiterate testers will prove more
difficult. Thankfully, one of our team members was able to get their
parents to help out; less than ideal, as that particular user group
will be under represented, but probably sufficient.

Nielsen also stresses the importance of focusing on what users do, not
just on what they say: they may say they didn't struggle to find an
element on the page, but their eye and mouse movements tell a
different story. In the absence of the necessary hardware and software
to monitor eye and mouse movements we instead opted to carry out the
tests in pairs. This way one conductor can focus on the user and the
other on the screen.

However, we can still use what the user has to say about the system,
and we used the same rating system as in the CHE. This provides a
solid metric with which we can identify key problems with our system.

\subsection{TBUE Data}

\begin{tbuetable}{Task 1: Find challenging recipes}

  1 & Expected filter to apply after pressing enter, but filter button
  present & 1 \\

  1 & Felt filter button could stand out more, felt it was small but
  also said it was easy to find as it was where they expected it to be &
  0 \\

  1 & Skipped over welcome page tutorial, then wasn't sure which view
  type to use for experienced cook, cycled through all the views until
  they found one they liked. & 2 \\

  2 & Read welcome page thoroughly, commented not all information was
  needed, felt that it slowed them down a bit & 1 \\

  2 & Completed task with no problems, initially drawn to list of
  recipes but found filters quickly & 0 \\

\end{tbuetable}

\begin{tbuetable}{Task 2: View a Vanilla Slice recipe in its most
    verbose form}

  3 & Not obvious what the different views are---descriptions not very
  obvious, wasn’t sure what the most verbose view is, content layout is
  fine & 1 \\

  3 & Ingredients list not formatting fractions properly, hard to read
  but can still make out values & 1 \\

  3 & Asked to choose default view, just want to see recipes, slows
  user down, welcome section too prominent ignore the search box & 2
  \\

  3 & Cosmetic issue title not being capitalised & 0 \\

  3 & Search bar not obvious, user scrolled first before choosing to
  search, but they did find the recipe complained that it took a bit
  longer & 2 \\

\end{tbuetable}


\begin{tbuetable}{Task 3: Find all vegetarian dishes that can be
    prepared in less than 20 minutes}

  3 & Search vegetarian, didn’t return anything, filters are there so
  can recover & 2 \\

  4 & Felt welcome page was not relevant to their task or scenarion,
  slowed them down & 1 \\

  4 & Filter by vegetarian gives nothing, no information to suggest if
  search return was empty or website errored, assumed website had broken
  and stopped & 4 \\

\end{tbuetable}

\begin{tbuetable}{Task 4: Find all pie recipes}

  5 & Pie not left in the search bar, even though title of page says
  ``you searched for pie'' inconsistency with pie not displayed in top
  search bar & 1 \\

  6 & Completed task with no problems & 0 \\

\end{tbuetable}

\begin{tbuetable}{Task 5: Change the default recipe view}

  7 & No problems reported, went back to page they orginally saw & 0 \\

  8 & Nothing on page to show how they could change it, guess its in
  help, help doesn’t help doesn’t mention default view. Doesn't know
  what th default view is or how to change it. & 4 \\

  8 & Stumbled upon answer by going back to home page, and going through
  first user flow & 4 \\

\end{tbuetable}


\subsection{Issues Uncovered}

Based on the results of the task based evaluation we have chosen to
present problems relating to the tutorial features of our
website. This includes the welcome and help page, where the user is
told about the different view types and goes to set the default view
as well as be given other useful information on how to use the
website. The following are specific test cases that produced
interesting results.

\paragraph{Task: find vanilla slice recipe and view it in its most verbose form.}

The user successfully completed the task by navigating the tutorial
and then using the search bar to find the desired recipe. Afterwards
he did express some concern over finding the most ‘verbose’ recipe
view described by the task. He further elaborated that whilst he was
given a summary of all the view types in the tutorial he failed to
decipher which view the task was referring to based on the tutorial
text. Perhaps work needs to be done to change the tutorial text to
make it more obvious but the user did also confess to reading the
tutorial particularly thoroughly. The user did state he had no issues
with the layout of the tutorial, the text was obvious and easy to
read, but it seems the user was more concerned about actually finding
the recipe as such whilst the tutorial presented relevant information
the user described it as ‘slowing him down’. Further elaboration
reveals this user would have been happy to discover the different
views of his own accord, but did state he probably would want to get
this information at a later time. The user chose to give this a
problem rating of 2, but they also claimed a large part of the fault
lay in the ambiguity between the task and the website and so their
rating may be weighted a bit heavily. From our point of view this
purposeful disconnect provided valuable additional information on how
the tutorial system is perceived by this user.

\paragraph{Task: Change the default recipe view}

The user remembered that they were given the opportunity to change
default view from the tutorial on the home page. The opted to go
straight back to the home page and change the default from there. They
did however concede that they did not expect the tutorial to show up
every time but opted for that route after a quick scan of the recipe
page showed no obvious way to change the default.

Another user, however struggled with the task, they spent a long time
on the recipe page particularly around the view changing buttons to no
avail. they then attempted to go to help, nothing there was helpful in
fulfilling the task. Eventually they stumbled upon the home screen and
were presented with the tutorial. They strongly disliked the flow they
took and it did take a long time to complete the task, they decided to
rate it 4 a very serious problem.

\subsection{Changes}

\todo{This Section}

Apart from the bug fixes that are required the task based evaluation
has caused us to reconsider a few things regarding the design of our
website. Firstly, tutorials will be changed to only show on first use
of the website. This brings us in line with what our volunteers were
expecting. A direct result of this change is also changing the way
default recipe views are chosen. As the tutorial is no longer
presented after first use a new way must be added to change default
view.

Most likely this will be added to the recipe page, around the same
area as the buttons to change the view in order to stay in line with
principles regarding grouping. It should be noted that the ability to
reset the tutorial will also be added, being placed in the help
section. Now that we are only showing the tutorial once we don’t want
to completely remove it from the user if they do not wish it to be.
