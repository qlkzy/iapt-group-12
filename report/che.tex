\section{Collaborative Heuristic Evaluation}
Our group decided to follow the CHE methodology outlined by Petrie and Buykx in the paper “Collaborative Heuristic Evaluation: improving the effectiveness of heuristic evaluation”. Coming into the CHE session each group member was asked to generate a list of potential problems to present to the group. These were collated and as a group were shown in sequence. Each group member individually rated the problem using the scale shown in figure~\ref{che1}. Care was taken to not get drawn into a discussion of solutions in this session but to instead focus on identifying the problems. Petrie and Buykx suggest including a domain expert in this session, this was not needed as the development team were completing the evaluation. This did show a major flaw in our methodology in the sense none of the group members are particularly experienced user experience evaluators. Petrie and Buykx elude to the fact that we should use user experience experts guided by one or two domain experts, not the other way around and certainly not without any user experience experts. So while this problem is not really avoidable given the nature of our project it does throw into doubt some of the data we generate within this evaluation. Extra preparation will be needed on the part of our team to get as close to the expertise level of actual user experience experts as is realistically possible.

\subsection{Evaluation Heuristics}
Our group was given the choice between using Nielsen Heuristics for interactive systems or the Petrie and Power Heuristics for Interactive Web Applications. Even though the Nielsen heuristics are largely more popular we chose to use the Petrie and Power Heuristics. We felt more comfortable using the more fleshed out categories, and what little experience our group had with evaluating heuristics was with the Petrie and Power set. Seeing as the level of evaluation expertise of the group is a major concern it would seem foolish not to go with what we are experienced with and therefore throw away whatever small improvement of the reliability of data the produced by this exercise.

\subsection{Results}
The results for the CHE are contained within figure~\ref{che2}. The task that displayed the majority of severe heuristic issues are ones involving the recipe page, in particular we asked the user to read a recipe and navigate between the different view types. We have decided to highlight the issues that were granted a rating of 2.5 or above, even though all the issues will be tackled these are the highest priority changes and the ones with largest impact.
\begin{center}
\emph{hard to tell where one ingredient begins and the other ends} (see figure~\ref{che3})
\end{center}
This breaks heuristic 8, provide clear well organised information structures. The information is all present however it is very difficult to see the boundaries between items in the list. The structure does not suggest that the information is a list and as such the users first instinct is to read it as a paragraph, for which it is not grammatically suitable. After reading the paragraph it becomes obvious the information should be presented as a list allowing the user to recover. With an average rating of 2.75 this problem does not stop the user completing their task but in our opinion will definitely slow them down and provide a source of irritance. Coupled with the fact that the recipe list is likely to referenced often during the process of reading a recipe, this problem is a significant flaw in our prototype.
\begin{center}
\emph{no indicator recipe instruction can be checked off} (see figure~\ref{che4})
\end{center}
Here we are guilty of breaking interactivity heuristics “how and why” and “interactive elements should be clearly distinguished”. Though we do provide some feed forward when the cursor is moved over the element we rely completely on the users curiosity, and a certain amount of luck, for the user to discover this functionality. At some point we must tell the user that our site allows them to check off items from the list or provide some visual indicator that the recipe list is not just a list of text but a list of pressable elements. With an average severity rating of 3.5 we clearly see this as a major issue, unless something is done the user is going to completely pass this functionality by.
\begin{center}
\emph{no idea which view you are in/press button for view that you are in, nothing happens} (see figure~\ref{che5})
\end{center}
This does not provide “clear labels and instructions” in relation to both interactive elements and the current presented content and fails to “provide feedback on user actions and system progress”. The user is not able to definitively know which view they are in, short of figuring it out based on the content which if they are the novice user isn’t going to happen. Related to this problem, the fact that nothing happens if you press the button for the view you are currently on is a cause for concern. To the user there is no difference between this button and the others and as such it is reasonable for them expect something to happen when they press it. When they press it, and nothing happens, there is a complete lack of feedback. The user is not given any indication that their button press has been registered let alone what effect it had on the system. 

Both these problems were rated very highly both having an average over 3, though they are serious problems that affect the usability of the site it is possible for the user to user their problem solving skills to complete their task. Thus we decided it did warrant the catastrophic rating even though we will definitely be addressing this issue.

\begin{center}
\emph{changing view: don't know what view change buttons fo or what they relate to/don't know what other views are} (see figure~\ref{che5})
\end{center}
There is a problem with the buttons that change from the different types of recipe view. There is no explanation for what these different views are or the effect they have on the recipe view by switching between them. This breaks the “how and why” heuristic, obviously the problem lies in not educating the user to what these options are, they could figure out what they do by trial and error and such this problem was rated 3.5/2.5 but again not quite a 4. 

\begin{center}
\emph{no distinction between instructions} (see figure~\ref{che4})
\end{center}
Again we are breaking the information architecture heuristic of “providing clear, well organised information structures”. Very similar to the problem with the ingredients list each step of the recipe is an element in a list but it is presented like one complete body of text. Not a problem in narrative view as it is mostly one large piece of text but a major issue in step by step and segmented views as the main purpose of the view is to break down a lengthy recipe into steps. The group rated the problem at 2.5, though it slows down the user considerably and is not pleasant to use, the user can still ultimately carry out their goal.

\subsection{Application Design Changes}
The first part of the redesign addresses the problems with “providing clear, well organised information structures”, namely the ingredients and recipe instruction lists now have clear spacing between elements within them (see figure~\ref{che6}). Care was taken to ensure the grouping remains intact so that the user can easily tell what is part of the instructions and what is part of the ingredients. To aid this distinction further a different line spacing style was used in both elements, the ingredient list is slightly tighter packed so that it is obvious that text does not continue on from the recipe step to whichever ingredient is level with it.

The larger change is related to the changing view type buttons. The positioning was moved to above the ingredients and recipe instructions. As changing the view changes the way these elements are displayed we thought it best to form a grouping between these and the buttons that control them to suggest to the user that they are related. We are now using a segmented control style of button, the view we are on will now have a visual indicator that it is currently selected (see figure~\ref{che7}). This solves both the problem of the user not knowing what view they are currently on as well as allowing us to disable pressing of the button for the view that is already displayed.

The problem of informing the user what the different view types are and what they do was more difficult to handle. We thought of adding extra labelling to the buttons but felt it would create undue clutter. Instead we opted for a one time tutorial (see figure~\ref{che8}). Either taken upon first visit of the website, where everything will be explained, or if this is declined a shorter version presented when first viewing a recipe. Ultimately we thought this the best way to inform the user of what they different views are and why they might want to use them, the information that needs to be communicated is too complicated to fit on the button structure and only need be shown once and so fits in well with a tutorial system. The user can call up the tutorial again by going through help. 

Similarly this tutorial will also be used to tell the user about the sites functionality to check off steps in the recipe. Mousing over a step will cause it to display some feedforward to further cement the idea.
