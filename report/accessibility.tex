\section{Accessibility Evaluation}

\subsubsection{Tools}

HTML CodeSniffer

\subsection{Validation}

\subsubsection{Evaluation Methodology}

\begin{itemize}
\item HTML CodeSniffer
\item W3C Validator
\end{itemize}

\subsubsection{Issues Found}

\subsection{Semantic Markup}

\subsubsection{Evaluation Methodology}

\begin{itemize}
\item Inspection
\item HTML CodeSniffer
\end{itemize}

\subsubsection{Issues Found}

\subsection{Alternatives}

\subsubsection{Evaluation Methodology}

\begin{itemize}
\item HTML CodeSniffer
\end{itemize}

\subsubsection{Issues Found}

\subsection{Labels}

\subsubsection{Evaluation Methodology}

\begin{itemize}
\item HTML CodeSniffer
\end{itemize}

\subsubsection{Issues Found}

\subsection{Keyboard}

\subsubsection{Evaluation Methodology}

\begin{itemize}
\item Tab through elements, checking for ordering and that all
  elements are reachable.
\end{itemize}

\subsubsection{Issues Found}

\begin{itemize}
\item Feedback on keyboard selection in header
\item Search page.
\item Feedback on keyboard selection on recipe page
\end{itemize}

\paragraph{Highlighting of focused elements.}
Several interactive elements with more complex visual styling---in
particular, the buttons in the header bar and for selecting the level
of detail at which to show recipes---had, in the process, lost the
builtin visual feedback showing which element was selected for
keyboard navigation. This is ironic, as much of the visual styling of
these elements existed to provide more feedback for mouse-based
navigation.

These issues were generally easy to fix using the \verb!:focus!
  pseudo-selector to replicate the browser built-in focus feedback
  (with increased emphasis in some cases to cut through the visual
  complexity).

\paragraph{Keyboard activation of extra interactive functionality.}
Richer pieces of interaction further from the original
hypertext-document model of the web web (in particular, the popup help
and tickable instructions) inherited no default keyboard-navigability
from the browser, and were for some time impossible to control from
the keyboard. This is particularly embarrasing given that the author
of this functionality uses a keyboard-oriented browser with an
interface borrowed from the \textsc{Unix} editor \verb!vim! for
regular browsing.

These issues were more problematic to resolve, as the \verb!onclick!
  event (also fired by keyboard activation) is not consistently bound
  to keyboard activation except in the case of \verb!a!,
  \verb!button!, and \verb!input! elements. This required a few
  slightly irritating changes to the markup and styling of the
  application.

\paragraph{Tab order of search filters}

Search filters sit naturally on the right, for visual effect and
inter-application consistency.

`Natural' tab order for search filters is therefore after all search
results; this is really inconvenient.

Placing search filters earlier in tab order is easy partial fix, but
causes focus to caper confusingly around the page.

Full resolution to place filters earlier in tab order and to use
script and styling to strongly emphasise the filter box when filters
are selected, minimising confusion.

\subsection{Feedforward and Feedback}

\subsubsection{Evaluation Methodology}

Ad-Hoc - list of interactions?

\subsubsection{Issues Found}

\begin{itemize}
\item Ticking off instructions
\item Setting default view
\item Search filters?
\end{itemize}
