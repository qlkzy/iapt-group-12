\documentclass{article}
\usepackage[utf8]{inputenc}

\title{IAPT Group 12 - Personae and Scenarios}
\date{SPR/SUM 2014}

\begin{document}

\maketitle
\clearpage

\section*{Personae}
\subsection*{Jennifer}
Jennifer is a first-year student in History at the University of York. She is 
living on her own for the first time, and after spending the first term eating
primarily frozen dinners and instant noodles, she has decided it is time to 
learn how to cook for herself. Jennifer is computer-literate, and in fact quite
savvy on the web, but has little experience with kitchen appliances or utensils.

\subsection*{Dale}
Dale is a 45 year-old headmaster at a primary school. He has been cooking at home
since he was about nine years old, being able to make complex meals for dinner
parties since he was sixteen. He revels in trying out new recipes and the more
complex the better. He regularly uses an iPad at home, including in the kitchen
where he has access to all of the latest cookery apps and blogs.

\subsection*{Bob}
Bob is a 16 year-old student. He aspires to become a professional cook after he 
finishes school and experiments with cooking as much as he can. He often prepares
dinners for his family and sometimes even his friends. Bob likes finding new
recipes on the internet and always looks for meals that would test his abilities.

\subsection*{Mike}
Mike is a 52 year-old store clerk. He is recently divorced and didn't really have
to worry about cooking until this point in his life. Mike can't afford to eat
take-away food anymore so he is trying to learn how to cook for himself. He knows
his way around a computer, but is not completely confident with technology.

\subsection*{Mavis}
Mavis is a 65 year-old retiree, living alone. She has been cooking the same set 
of meals with a high-level of skill for the majority of her life, and has
decided that now she has a lot of time on her hands she would like to develop
her repertoire. She would also like to use this time to get up to speed with 
mobile technology, having never owned so much as a cell phone in the past.

\subsection*{Andy}
Andy is a young man in his early twenties, who is comfortable in the kitchen but
mostly cooks simple and straightforward food. He dislikes cooking from recipe
books, but resorts to them for more complex meals and those with which he is
unfamiliar. He tends to speed-read recipes, often skipping steps or executing
instructions in the 'wrong' order, but is normally able to 'wing it' and produce
something reasonable.

\clearpage

\section*{Scenarios}
\subsection*{Novice Cook - Jennifer}
Jennifer decides to cook a nice but fairly easy meal for herself and her 
housemates. Earlier in the day, she browses through the recipe website, looking
for fairly simple-looking recipes with cheap and easily-available ingredients.
Then, she goes to her local supermarket, navigates to the recipe on her phone,
and walks round picking up ingredients, checking off items in the recipe as she
buys them. She then checks how long it will take to cook the recipe, so she can
start at the right time, and heads off to a lecture.

That evening, she sets up her laptop in the kitchen to look at the recipe,
the easiest-to-follow view of the recipe, and starts cooking. She starts by 
measuring out all the ingredients she will need, and then starts working through
the individual steps. At one point, her housemates come into the kitchen and 
start up a conversation. She joins in, and when she returns to cooking realises
that she has forgotten where she was in the recipe. Fortunately, she was
checking off recipe steps as she did them, and the recipe site helps her find
where she left off to continue cooking.

\subsection*{Checking Ingredients - Dale}
Dale is cooking Quiche Maritime, one of his favourite recipes, which he has been
too busy to cook for a while. He starts cooking, and then realises that, while
he's pretty sure the recipe needs three eggs, he can't remember which part of the
recipe calls for two of the eggs, and which for one. He quickly navigates to the
recipe site on his tablet, and searches for Quiche Maritime. Finding the recipe,
he quickly scans it for places where eggs are mentioned, and works out when to
use the various eggs. He then goes back to cooking, without needing to refer to
the recipe again.

\subsection*{Dinner Party - Dale}
Dale is planning a dinner party and would like ideas on what to cook. One of his
guests is vegetarian and so Dale would like to make sure that at least on of the
dishes that he prepares is suitable for vegetarians. He would also like to avoid
shopping for ingredients if possible, and so would like to know how he can make
his chosen meal with ingredients that he currently possesses (via substitution).
He also needs to know (roughly) how long the meal will take to complete as he 
needs to be ready in time for his guests' arrival.

\subsection*{Helping Others - Dale}
Dale is helping his daughter at university cook a meal. He would like to send her
to the app to find a recipe that he suggests. He wants to check that it is easy 
to make and would preferably like to search for a recipe based on his daughter's
specific tastes. He then wishes to check that she has the required equipment and
ingredients and also that it will not take her long to cook. He also knows that 
his daughter is quite impatient, and will likely make mistakes because of this.
He would therefore like concise, step-by-step instructions with the option to
expand the detail if needed.

Dale's daughter calls him half-way through cooking to ask for help. Dale would 
like to be able to quickly find the recipe that she is working on, and jump to 
the step at which she is stuck in order to help her.

\subsection*{Missing Ingredients - Andy}
Andy wants to cook a potato salad, but has no lemon or garlic to make the
gremolata. He navigates to the recipe app, and starts cooking in the narrative
view.

Partway through, he discovers that he's having trouble keeping track of what he
should be doing, so he switches the recipe to the step-by-step view, and
discovers that he has done two of the steps in a different order than the 
step-by-step view prescribes (the narrative view didn't specify an order). He
checks off the out-of-order step, and then continues with the recipe in the
order given, skipping the steps relating to the gremolata.

\subsection*{Difficulty - Bob}
Bob has decided to make a meat pie for his grandmother. On his way back from 
school he picks up the groceries that he thinks he'll need for the pie. Bob finds
a suitable recipe using the app. The app allows him to choose his expertise level
and informs him that it affects how the recipe is displayed. He chooses 'expert'
at first, but midway through Bob realizes that it's a bit too hard to follow and
decides he would like to change to the 'intermediate' view. After the change, he
has no trouble finishing the meat pie.

\subsection*{Losing Place - Mike}
Mike decides to make an omelette one evening for himself. He finds a fairly 
simplistic, easy to follow recipe using the app and ther goes to a store to get
the required ingredients. When he comes back home, he realizes that he absent
mindedly shut off his device running the app, however when he loads up the 
app after booting up the device he notices that the app has recorded his 
recent activity and he has no trouble finding the omelette recipe again. He
follows the instructions and successfully finishes preparing the dish.

\subsection*{First Time With Tech - Mavis}
Mavis has decided to purchase a tablet to improve her relationship with modern
technology. One of the first apps that she asks her daughter to install for her
is the recipe app, so that she can get started with trying out some new recipes
without the need to pore over lots of cookery books. Mavis is completely 
unfamiliar with interacting with tablet devices, and would like an interactive 
tutorial process to 'show her the ropes' that doesn't appear to patronize her.

\end{document}