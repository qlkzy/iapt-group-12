\section{Design Rationale}

\subsection{System Overview}
The final recipe-presentation web application was the product of a many-step process. These steps, in order, were:
\begin{itemize}
  \item Research into similar existing services, carried out to provide insight into the downfalls and successes of applications produced by others;
  \item Analysis of the application specification;
  \item Devising of user personas and scenarios for those personas, created to offer something of a user-centred perspective to the design process;
  \item Formation of formal requirements for the application, these providing an easily-verified set of criteria laying out the minimum functionality required for the application;
  \item Creation of an interactive first prototype;
  \item Collaborative heuristic evaluation of the first prototype;
  \item Redesign of the prototype to rectify problems uncovered in the CHE, and to bring the prototype more in line with both functional requirements and the goals implicit in the user scenarios;
  \item Task-based user evaluation of redesigned prototype;
  \item Final adjustments of the protoype to alleviate issues raised in the TBUE, and to ensure that goals and functional requirements were met;
\end{itemize}

From the initial reasearch into other services and analysis of the specification, many personas and scenarios were created. The purpose of these was to both identify the potential users of the system to be designed and to aid in the formation of formal functional requirements. The personas and scenarios follow.
\begin{center}
\begin{tabularx}{\textwidth}{|X|X|X|}
\hline
Name & Details & Scenario \\ \hline
Jennifer & Female, early-twenties, novice cook, computer-literate. & Starts cooking, becomes distracted. Comes back to checked off recipe steps. \\ \hline
Dale & Male, middle-aged, accomplished cook, computer literate. & Hosting dinner party, needs vegetarian dish, preparation needed to time guests' arrival. \\ \hline
Bob & Male, teenager, inexperienced but keen cook, computer literate. & Begins viewing a recipe in narrative, finds it too difficult, changes to segmented. \\ \hline
Mike & Male, middle-aged, no cooking experience, computer illiterate. & Shopping for ingredients, forgot to make a list, needs a list of ingredients for recipe. \\ \hline
Mavis & Female, elderly, experienced cook, computer illiterate. & Wants to use her recently purchased tablet to look up recipes. \\ \hline
Andy & Male, early-twenties, intermediate cook, computer literate & Begins using narrative view, switches to step-by-step for more information, finds and checks off the step he was at. \\
\hline
\end{tabularx}
\end{center}
Andy: young male, intermediate cook and computer-literate.
Scenario: Begins using the narrative view but switches to step by step for more information. Finds and checks off step he was on before continuing.

Following the personas and scenarios came the functional requirements. These were laid out to provide clear-cut goals which needed to be achieved by the application in order for it to provide a satisfactory quality of service. These requirements were split into two headings, each heading reflecting an essential component in the interactive system:

\paragraph{Recipe Browsing}
Users must be able to:
\begin{itemize}
\item search for recipes using a full or partial recipe name;
\item view a list of all recipes that are present in the application's database of recipes;
\item filter any list of recipes using specific filtration criteria.
\end{itemize}

\paragraph{Recipe Viewing}
Users must be able to:
\begin{itemize}
\item see general information about a recipe both when browsing multiple recipes and when viewing a single recipe;
\item see the ingredients required to complete a recipe;
\item see the instructions that need to be followed in order to complete a recipe;
\item change the level of detail in which recipe instructions are presented to them;
\item set the default level of detail in which all recipe instructions are presented to them;
\item mark instructions as completed through some interation.
\end{itemize}

Although all of the functional requirements were met in the final version of the application, the functionality of the system that has been produced is a little richer than the functionality suggested by the functional requirements. The complete functionality of the system consists of that functionality contained in the functional requirements plus the following: 

\section*{Rating Usability of User Interfaces}
Out group attempted to distill and combine Shneiderman's Principles of Interaction Design, Don Norman's Design Principles and Tog's First Principles of Interaction Design, in order to make them easier to work with. These criteria have be identified as a solid starting point for any user interface.

\subsection*{Design Criteria}
\begin{itemize}
\item 1. Consistency;
\item 2. Informative feedback to the user;
\item 3. Straightforward reversal of actions;
\item 4. Promoting internal locus of control;
\item 5. Low short-term memory load;
\item 6. High feature visibility;
\item 7. Information readability;
\end{itemize}

\subsection{Viewing a Recipe}
The following is the design rationale behind the first piece of complex functionality found in the produced application: the process of viewing a recipe.

When designing the recipe view we tried to maintain a strong sense of consistency to the page, as identified in criteria 1, all fonts are uniform within their respective page groups (see figure~\ref{dr1}). We decided against bolding the names of ingredients in the recipe instructions even though we do bold the ingredient amounts. We felt this would just introduce unnecessary clutter breaking the sense of consistency and perhaps provide a false feedforward suggesting to the the user that the name is pressable when it is not.

Similarly we maintained information readability, criteria 7, by using consistent and readable font sizes and spacing. Our primary rationale behind this coming from the personas and scenarios,in particular Mavis. Not only is readability an issue for elderly (due to potentially reduced eye sight) but is also a major issue for users on mobile devices. As her scenario outlines her recent purchase of a tablet, we decided to pay extra attention to readability to ensure our website be usable on mobile without us having to create a separate web client.

The only exception is the difference in spacing between ingredients and instructions. As discussed in Jennifers scenarios (and others) we envisage users wanting to use the site before they are ready to cook. We believe that they will be looking up ingredients for dishes while shopping. It is for this reason we purposely break down the readability between the two to cement them as separate elements to the page making it easier to quickly look up the ingredients.

Short-term memory load, criteria 5, was identified as a key issue early on, as outlined in Mike's persona part of our user base are likely to be very novice cooks. The may well be using the website whilst cooking and will need information delivered to them in digestible chunks so as to not overwhelm them. We tried to ensure this by providing adequate spacing between instructions especially within the step by step view (see figure~\ref{dr2}). The separation encourages the user to read a step and then process the information before moving on to the next. We feel the spacing is not enough to make the user think it is not belonging to the same group and at the same time is sufficient enough to ensure low short-term memory load. The User may well be dipping in and out of reading the page and thus giving them information in small chunks will be beneficial, this is shown through Andy's scenario where he refers to the step by step view while cooking. The separation means he can easily scan the separate steps to find the one he should be on.

On way in which criteria 6, high feature visibility, was upheld was by ensuring the buttons to change the type of view are coloured differently to their surrounding elements (see figure~\ref{dr2}). Though their relative size is small we feel the colour difference is sufficient enough to draw attention to it without making it the main feature of the page.

Regular users of our website, users potentially like Bob, are likely to be accessing recipes often. We want to provide them with a sense of customisability and control over their experience. By allowing them to choose which type of recipe view the website defaults to we hope to promote internal locus of control. To this effect the view buttons also allow the user to set a default view (see figure~\ref{dr2}) and thereby fulfilling principle 4.

Criteria 3 provided an interesting dilemma for our design. The type buttons do not immediately provide straightforward reversal of action for the user, pressing a view button twice will not undo its action. We did this on purpose to stay inline with what our research of other websites has shown us, we think the user will not expect this behaviour and as the other buttons are in close proximity they will use those to undo the action. Though our more tech illiterate users may struggle with this, Mavis not being familiar with established website design patterns as outlined in her persona, we believe the majority of our user base will appreciate it being done this way. As Mavis becomes more familiar with internet usage this will not be a problem for her either, and in fact we should follow the norms of other segmented control patterns so as not to confuse her when she goes to other website.  

A more concrete way in which criteria 3 is followed was in the design of the instructions ticking off feature. We do provide an easy reverse for ticking instructions, as users would expect pressing the element again reverses the action. The ticking off feature has also been designed to have high feature visibility. Empty tick boxes are always present which already suggest to the user that they can be ticked off, but further to this when a step is moused over it is highlighted providing the extra feedforward to further clarify its functionality.

\subsection{Filtering Lists of Recipes}
The following is the design rationale behind the second piece of complex functionality found in the produced application: the process of filtering lists of recipes.

After examining the personas and scenarios created by our group we came to the conclusion that easy recipe browsing and access is a recurring theme and decided that it is a high priority issue. Most cooking websites have a high volume of recipes, therefore robust tools are needed. Our solution to this problem was the introduction of the filtering menu. The multiple available choices enable the site visitors to get especially specific with their recipe needs. It shows up whenever user is browsing the recipes or after performing a search.

While designing the filters we used Shneiderman’s Eight Golden Rules of Interface Design and Don Norman’s design principles.

Based on our research of other recipe websites and phone applications available on the web as well as our scenarios we distilled the following available filter types and their selections were chosen as follows. First filter is Dish type, with available selection of Main, Salad, Side and Dessert. In our scenarios, personas are cooking for very different occasions. The loner Mike might be trying to get a simple main course dish done for himself, while Dale might desire to prepare a dessert for his friends.  Next is Dietary restriction, which allows choosing between Vegetarian, Vegan, Kosher and Halal. We considered this especially important, since it caters to users with extremely varying tastes. For example, in one of our scenarios Dale is preparing for a dinner party. One of his friends is a vegetarian, therefore Dale wants to make sure that at least one of the dishes he prepares is vegetarian, so that all of his guests are satisfied. Difficulty filter has the expected choices of Beginner, Intermediate and Advanced, making it easy for the site visitors to narrow down the selection to their desired skill level for the recipes. These choices stem from our created personas such as Mike, who does not have any cooking experience whatsoever and struggles with even the most basic recipes, and Bob, who is on the opposite end of the cooking spectrum, and prefers to use recipes that would challenge his abilities and help him improve as a cook.  

The serving information allows the user to specify the number of portions that can be made with the recipe. This filter can be justified by our scenarios and personas. For example, Mike, cooks only for himself, since he lives alone, he only looks for recipes that have 1 serve. Another example is Jennifer, in one of the scenarios she is cooking a meal for all her housemates and wants to make sure that there is enough food for everyone. The Time Up to field allows to choose the maximum time a recipe would take to prepare. This is useful to one of our personas, Dale, who is cooking for his friends and wants to be done with cooking before they come over. The time and serves fields have placeholder text, further explaining what those fields accomplish in regards to the filtering system. This acts as an affordance and helps the user figure out what inputs are expected from him or her as can be seen in figure~\ref{dr3}

The filter menu and the recipes use the same gradient background so that the user would naturally associate the two. The filters work and look exactly the same on both browse page as well as on search results page (except the filtering is done on the search results, rather than the whole database). This provides expected consistency between the pages, which means that the user has to figure out how to use the filters only once.

The filter menu takes up prominent space in the page to immediately draw attention to it. This is done with the intention of making sure that the user does not miss this important part of the website. This acts in accordance with Don Norman’s principles of high feature visibility and our criteria 6.

The reset button takes back the filter to its default state as well resetting the the recipe results, providing easy reversible actions as suggested by criteria 3. Both the filter and the reset buttons are made to look obviously clickable, which acts as an affordance as described in Don Norman’s Design Principles. The filters immediately react to changes by the user, offering immediate feedback on the performed operation and fulfilling criteria 2. This also supports the internal locus of control rule described by Shneiderman and us in criteria 4. After a filtering operation is performed, the filter menu choices previously made are still visible in the menu. This allows for easy adjustments and also reduces the short term memory load (criteria 5) while browsing the recipes, because the user does not have to remember the choices made previously.

If after filtering there are no recipes found a helpful message shows up saying that the search came up with no results as illustrated in figure~\ref{dr4} again providing informative feedback and fulfilling criteria 2. This means that the user understands what’s happening and shows that the website is still functioning correctly. This is in accordance with the 3rd rule ‘Offer Informative Feedback’ from  Shneiderman’s Eight Golden Rules of Interface Design. 

The results themselves are displayed in the middle of the screen and are designed to be the centerpiece of the website. There is clear separation between individual results while still maintaining visual consistency in accordance with criteria 1 and 7. The recipes are displayed as information cards, ideally providing enough information at a glance to make an informed choice. This was inspired from our research done on popular cooking web sites available online. Each recipe is clearly separated from each to avoid confusion, however there is still visual consistency amongst them. Clicking on a recipe card leads to the full view of a recipe. To make this obvious, the recipe cards were designed to look like clickable buttons.
