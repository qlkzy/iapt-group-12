\section{Design Rationale}

\subsection{Personas \& Scenarios}
From both the initial reasearch into other services and group discussion about the specification, many personas and scenarios were devised. The purpose of these was to both identify the potential users of the system to be designed and to aid in the formation of formal functional requirements. The personas and scenarios follow.

\paragraph{Jennifer}
\subparagraph{Persona}
Jennifer is a first-year student in History at the University of York. She is 
living on her own for the first time, and after spending the first term eating
primarily frozen dinners and instant noodles, she has decided it is time to 
learn how to cook for herself. Jennifer is computer-literate, and in fact quite
savvy on the web, but has little experience with kitchen appliances or utensils.
\subparagraph{Scenario}
Jennifer decides to cook a nice but fairly easy meal for herself and her 
housemates. Earlier in the day, she browses through the recipe website, looking
for fairly simple-looking recipes with cheap and easily-available ingredients.
Then, she goes to her local supermarket, navigates to the recipe on her phone,
and walks round picking up ingredients, checking off items in the recipe as she
buys them. She then checks how long it will take to cook the recipe, so she can
start at the right time, and heads off to a lecture.

That evening, she sets up her laptop in the kitchen to look at the recipe,
the easiest-to-follow view of the recipe, and starts cooking. She starts by 
measuring out all the ingredients she will need, and then starts working through
the individual steps. At one point, her housemates come into the kitchen and 
start up a conversation. She joins in, and when she returns to cooking realises
that she has forgotten where she was in the recipe. Fortunately, she was
checking off recipe steps as she did them, and the recipe site helps her find
where she left off to continue cooking.

\paragraph{Dale}
\subparagraph{Persona}
Dale is a 45 year-old headmaster at a primary school. He has been cooking at home
since he was about nine years old, being able to make complex meals for dinner
parties since he was sixteen. He revels in trying out new recipes and the more
complex the better. He regularly uses an iPad at home, including in the kitchen
where he has access to all of the latest cookery apps and blogs.
\subparagraph{Scenario}
Dale is planning a dinner party and would like ideas on what to cook. One of his
guests is vegetarian and so Dale would like to make sure that at least on of the
dishes that he prepares is suitable for vegetarians.He also needs to know (roughly) 
how long the meal will take to complete as he needs to be ready in time for his 
guests' arrival.

\paragraph{Bob}
\subparagraph{Persona}
Bob is a 16 year-old student. He aspires to become a professional cook after he 
finishes school and experiments with cooking as much as he can. He often prepares
dinners for his family and sometimes even his friends. Bob likes finding new
recipes on the internet and always looks for meals that would test his abilities.
\subparagraph{Scenario}
Bob has decided to make a meat pie for his grandmother. On his way back from 
school he picks up the groceries that he thinks he'll need for the pie. Bob finds
a suitable recipe using the app. The app allows him to choose his expertise level
and informs him that it affects how the recipe is displayed. He chooses 'expert'
at first, but midway through Bob realizes that it's a bit too hard to follow and
decides he would like to change to the 'intermediate' view. After the change, he
has no trouble finishing the meat pie.

\paragraph{Mike}
\subparagraph{Persona}
Mike is a 52 year-old store clerk. He is recently divorced and didn't really have
to worry about cooking until this point in his life. Mike can't afford to eat
take-away food anymore so he is trying to learn how to cook for himself. He knows
his way around a computer, but is not completely confident with technology.
\subparagraph{Scenario}
Mike decides to make an omelette one evening for himself. He finds a fairly 
simplistic, easy to follow recipe using the app and ther goes to a store to get
the required ingredients. When he comes back home, he realizes that he absent
mindedly shut off his device running the app, however when he loads up the 
app after booting up the device he notices that the app has recorded his 
recent activity and he has no trouble finding the omelette recipe again. He
follows the instructions and successfully finishes preparing the dish.

\paragraph{Mavis}
\subparagraph{Persona}
Mavis is a 65 year-old retiree, living alone. She has been cooking the same set 
of meals with a high-level of skill for the majority of her life, and has
decided that now she has a lot of time on her hands she would like to develop
her repertoire. She would also like to use this time to get up to speed with 
mobile technology, having never owned so much as a cell phone in the past.
\subparagraph{Scenario}
Mavis has decided to purchase a tablet to improve her relationship with modern
technology. One of the first apps that she asks her daughter to install for her
is the recipe app, so that she can get started with trying out some new recipes
without the need to pore over lots of cookery books. Mavis is completely 
unfamiliar with interacting with tablet devices, and would like an interactive 
tutorial process to 'show her the ropes' that doesn't appear to patronize her.

\paragraph{Andy}
\subparagraph{Persona}
Andy is a young man in his early twenties, who is comfortable in the kitchen but
mostly cooks simple and straightforward food. He dislikes cooking from recipe
books, but resorts to them for more complex meals and those with which he is
unfamiliar. He tends to speed-read recipes, often skipping steps or executing
instructions in the 'wrong' order, but is normally able to 'wing it' and produce
something reasonable.
\subparagraph{Scenario}
Andy wants to cook a potato salad, but has no lemon or garlic to make the
gremolata. He navigates to the recipe app, and starts cooking in the narrative
view.

Partway through, he discovers that he's having trouble keeping track of what he
should be doing, so he switches the recipe to the step-by-step view, and
discovers that he has done two of the steps in a different order than the 
step-by-step view prescribes (the narrative view didn't specify an order). He
checks off the out-of-order step, and then continues with the recipe in the
order given, skipping the steps relating to the gremolata.

\subsection{Functional Requirements}
The functional requirements arrived upon through the above work are as follows:

\subsubsection{Recipe Browsing}
Users must be able to:
\begin{itemize}
\item search for recipes using a full or partial recipe name;
\item view a list of all recipes that are present in the application's database of recipes;
\item filter any list of recipes using specific filtration criteria.
\end{itemize}

\subsubsection{Recipe Viewing}
Users must be able to:
\begin{itemize}
\item see general information about a recipe both when browsing multiple recipes and when viewing a single recipe;
\item see the ingredients required to complete a recipe;
\item see the instructions that need to be followed in order to complete a recipe;
\item change the level of detail in which recipe instructions are presented to them;
\item set the default level of detail in which all recipe instructions are presented to them;
\item mark instructions as completed through some interation.
\end{itemize}

\subsection{Viewing a Recipe}
The following is the design rationale behind the first piece of complex functionality found in the produced application: the process of viewing a recipe.

When designing the recipe view we tried to maintain a strong sense of consistency to the page, all fonts are uniform within their respective page groups (see figure~\ref{dr1}). We decided against bolding the names of ingredients in the recipe instructions even though we do bold the ingredient amounts. We felt this would just introduce unnecessary clutter breaking the sense of consistency and perhaps provide a false feedforward suggesting to the the user that the name is pressable when it is not.

Similarly we maintained information readability by using consistent and readable font sizes and spacing. Our primary rationale behind this coming from the personas and scenarios,in particular Mavis. Not only is readability an issue for elderly (due to potentially reducing eye sight) but is also a major issue for users on mobile devices. As her scenario outlines her recent purchase of a tablet, we decided to pay extra attention to readability to ensure our website be usable on mobile without us having to create a separate web client.

The only exception is the difference in spacing between ingredients and instructions. As discussed in Jennifers scenarios (and others) we envisage users wanting to use the site before they are ready to cook. We believe that they will be looking up ingredients for dishes while shopping. It is for this reason we purposely break down the readability between the two to cement them as separate elements to the page making it easier to quickly look up the ingredients.

Short-term memory load was identified as a key issue early on, as outlined in Mike's persona part of our user base are likely to be very novice cooks. The may well be using the website whilst cooking and will need information delivered to them in digestible chunks so as to not overwhelm the user. We tried to ensure this by providing adequate spacing between instructions especially within the step by step view (see figure~\ref{dr2}). The separation encourages the user to read a step and then process the information before moving on to the next. We feel the spacing is not enough to make the user think it is not belonging to the same group and at the same time is sufficient enough to ensure low short-term memory load. The User may well be dipping in and out of reading the page and thus giving them information in small chunks will be beneficial, this is shown through Andy's scenario where he refers to the step by step view while cooking. The separation means he can easily scan the separate steps to find the one he should be on.

We allowed for high feature visibility by ensuring the buttons to change the type of view are coloured differently to their surrounding elements (see figure~\ref{dr2}). Though their relative size is small we feel the colour difference is sufficient enough to draw attention to it without making it the main feature of the page.

Regular users of our website, users potentially like Bob, are likely to be accessing recipes often. We want to provide them with a sense of customisability and control over their experience. By allowing them to choose which type of recipe view the website defaults to we hope to promote internal locus of control. To this effect the view buttons also allow the user to set a default view (see figure~\ref{dr2}).

The type buttons do not immediately provide straightforward reversal of action for the user, pressing a view button twice will not undo its action. We did this on purpose to stay inline with what our research of other websites has shown us, we think the user will not expect this behaviour and as the other buttons are in close proximity they will use those to undo the action. Though our more tech illiterate users may struggle with this, Mavis not being familiar with established website design patterns as outlined in her persona, we believe the majority of our user base will appreciate it being done this way. As Mavis becomes more familiar with internet usage this will not be a problem for her either, and in fact we should follow the norms of other segmented control patterns so as not to confuse her when she goes to other website.  

We do provide an easy reverse for ticking instructions, as users would expect pressing the element again reverses the action. The ticking off feature has also been designed to have high feature visibility. Empty tick boxes are always present which already suggest to the user that they can be ticked off, but further to this when a step is moused over it is highlighted providing the extra feedforward to further clarify its functionality.

\subsection{Filtering Lists of Recipes}
The following is the design rationale behind the second piece of complex functionality found in the produced application: the process of filtering lists of recipes.

It was decided early on by our group that easy recipe browsing is a high priority for us. Most cooking websites have a high volume of recipes, therefore robust tools are needed. Our solution to this problem was the introduction of the filtering menu. The multiple available choices enable the site visitors to get especially specific with their recipe needs. It shows up whenever user is browsing the recipes or after performing a search.

While designing the filters we used Shneiderman’s Eight Golden Rules of Interface Design and Don Norman’s design principles.

Based on our research of other recipe websites and phone applications available on the web  we distilled the following available filter types and their selections were chosen as follows. First filter is Dish type, with available selection of Main, Salad, Side and Dessert. Next is Dietary restriction, which allows choosing between Vegetarian, Vegan, Kosher and Halal. We considered this especially important, since it caters to users with extremely varying tastes. Difficulty filter has the expected choices of Beginner, Intermediate and Advanced, making it easy for the site visitors to narrow down the selection to their desired skill level for the recipes. The serves allow the user to specify the number of portions that can be made with the recipe. The Time Up to field allows to choose the maximum time a recipe would take to prepare. The time and serves fields have placeholder text, further explaining what those fields accomplish in regards to the filtering system. This acts as an affordance and helps the user figure out what inputs are expected from him or her as can be seen in figure~\ref{dr3}

The filter menu and the recipes use the same gradient background so that the user would naturally associate the two. The filters work and look exactly the same on both browse page as well as on search results page (except the filtering is done on the search results, rather than the whole database). This provides expected consistency between the pages, which means that the user has to figure out how to use the filters only one.

The filter menu takes up prominent space in the page to immediately draw attention to it. This is done with the intention of making sure that the user does not miss this important part of the website. This acts in accordance with Don Norman’s principles.

The reset button takes back the filter to its default state as well resetting the the recipe results. Both the filter and the reset buttons are made to look obviously clickable, which acts as an affordance as described in Don Norman’s Design Principles. The filters immediately react to changes by the user, offering immediate feedback on the performed operation. This also supports the internal locus of control rule described by Shneiderman. After a filtering operation is performed, the filter menu choices previously made are still visible in the menu. This allows for easy adjustments and also reduces the short term memory load while browsing the recipes, because the user does not have to remember the choices made previously.

If after filtering there are no recipes found a helpful message shows up saying that the search came up with no results as illustrated in figure~\ref{dr4}. This means that the user understands what’s happening and shows that the website is still functioning correctly. This is in accordance with the 3rd rule ‘Offer Informative Feedback’ from  Shneiderman’s Eight Golden Rules of Interface Design. 

The results themselves are displayed in the middle of the screen and are designed to be the centerpiece of the website. There is clear separation between individual results while still maintaining visual consistency. The recipes are displayed as information cards, ideally providing enough information at a glance to make an informed choice. This was inspired from our research done on popular cooking web sites available online. Each recipe is clearly separated from each to avoid confusion, however there is still visual consistency amongst them. Clicking on a recipe card leads to the full view of a recipe. To make this obvious, the recipe cards were designed to look like clickable buttons.